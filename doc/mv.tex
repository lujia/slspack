
\chapter{Matrix and Vector}

\section{Matrix Definition}

SLSPack uses \verb|int| for integer and \verb|double| for floating-point number.
In this library, matrix indices and array indices follow C style, which start from 0.

\begin{evb}
typedef struct SLSPACK_MAT_
{
    double *Ax;
    int *Ap;
    int *Aj;

    int num_rows;
    int num_cols;
    int num_nnzs;

} SLSPACK_MAT;
\end{evb}

The definition of \verb|SLSPACK_MAT| is the same as standard definition.

\section{Matrix Management}

\subsection{Initialize}
\index{Matrix Init}

\vb{slspack_mat_init} initializes a matrix, which sets row, column and non-zero to zero and 
set arrays to \verb|NULL|.

\begin{evb}
void slspack_mat_init(SLSPACK_MAT *A);
\end{evb}

\subsection{Create}
\vb{slspack_mat_create} creates a CSR matrix using user input.
\begin{evb}
SLSPACK_MAT slspack_mat_create(int nrows, int ncols, int *Ap, int *Aj, double *Ax);
\end{evb}

\subsection{Destroy}

\vb{slspack_mat_destroy} destroys a matrix object and releases memory.
\begin{evb}
void slspack_mat_destroy(SLSPACK_MAT *csr);
\end{evb}

\section{Vector Definition}

\begin{evb}
typedef struct SLSPACK_VEC_
{
    double *d;
    int n;

} SLSPACK_VEC;
\end{evb}

\vb{SLSPACK_VEC} has two members, which are vector length (\vb{n}) and data (memory,
\vb{d}).

\section{Vector Management}

\subsection{Create}
\vb{slspack_vec_create} creates a length \vb{n} floating-point vector.
\begin{evb}
SLSPACK_VEC slspack_vec_create(int n);
\end{evb}

\subsection{Destroy}

\vb{slspack_vec_destroy} destroys a vector.
\begin{evb}
void slspack_vec_destroy(SLSPACK_VEC *v);
\end{evb}

\subsection{Set Value}
\vb{slspack_vec_set_value}, \vb{slspack_vec_set_value_by_array} and \vb{slspack_vec_set_value_by_index}
set vector values.

\vb{slspack_vec_set_value} sets the vector to the same value.
\begin{evb}
void slspack_vec_set_value(SLSPACK_VEC x, double val);
\end{evb}


\vb{slspack_vec_set_value_by_array} sets vector value by a buffer, which has the same length as
vector.
\begin{evb}
void slspack_vec_set_value_by_array(SLSPACK_VEC x, double *val);
\end{evb}

\vb{slspack_vec_set_value_by_index} sets value to the $i$-th component, $x[i] = val$.
\begin{evb}
void slspack_vec_set_value_by_index(SLSPACK_VEC x, int i, double val);
\end{evb}

\subsection{Get Value}
\vb{slspack_vec_get_value} copies vector's values to a buffer, which should have the same length as
the vector.
\begin{evb}
void slspack_vec_get_value(double *val, SLSPACK_VEC x);
\end{evb}

\vb{slspack_vec_get_value_by_index} gets the value of the $i$-th component.
\begin{evb}
double slspack_vec_get_value_by_index(SLSPACK_VEC x, int i);
\end{evb}

\subsection{Copy}
\vb{slspack_vec_copy} copies data from source to destination.
\begin{evb}
void slspack_vec_copy(SLSPACK_VEC des, const SLSPACK_VEC src);
\end{evb}
